\documentclass[11pt,a4paper,russian,intlimits]{ncc}
\usepackage[utf8]{inputenc}
\usepackage[T2A]{fontenc}
\DeclareUnicodeCharacter{2009}{\,}
\usepackage[margin=2cm]{geometry}

\usepackage{color}

\usepackage{hyperref}
\hypersetup{
    colorlinks=true,
    linkcolor=black,
    filecolor=magenta,
    urlcolor=cyan,
    linktoc=all,
}

\usepackage{graphicx}

% Needed for Asciidoc

\newcommand{\admonition}[2]{\textbf{#1}: {#2}}
\newcommand{\rolered}[1]{ \textcolor{red}{#1} }
\newcommand{\roleblue}[1]{ \textcolor{blue}{#1} }


\renewenvironment{quotation}
{   \leftskip 4em \begin{em} }
{\end{em}\par }

\def\signed#1{{\leavevmode\unskip\nobreak\hfil\penalty50\hskip2em
  \hbox{}\nobreak\hfil\raise-3pt\hbox{(#1)}%
  \parfillskip=0pt \finalhyphendemerits=0 \endgraf}}


\newsavebox\mybox

\newenvironment{aquote}[1]
  {\savebox\mybox{#1}\begin{quotation}}
  {\signed{\usebox\mybox}\end{quotation}}

\newenvironment{tquote}[1]
  {  {\bf #1} \begin{quotation} \\ }
  { \end{quotation} }

%% BOXES: http://tex.stackexchange.com/questions/83930/what-are-the-different-kinds-of-boxes-in-latex
%% ENVIRONMENTS: https://www.sharelatex.com/learn/Environments

\newenvironment{asciidocbox}
  {\leftskip6em\rightskip6em\par}
  {\par}

\newenvironment{titledasciidocbox}[1]
  {\leftskip6em\rightskip6em\par{\bf #1}\vskip-0.6em\par}
  {\par}



%%%%%%%%%%%%%%%%%%%%%%%%%%%%%%%%%%%%%%%%%%%%%%%%%%%%%%%%

%% http://texblog.org/tag/rightskip/


\newenvironment{preamble}
  {}
  {}

%% http://tex.stackexchange.com/questions/99809/box-or-sidebar-for-additional-text
%%
\newenvironment{sidebar}[1][r]
  {\wrapfigure{#1}{0.5\textwidth}\tcolorbox}
  {\endtcolorbox\endwrapfigure}


%%%%%%%%%%

\newenvironment{comment*}
  {\leftskip6em\rightskip6em\par}
  {\par}

  % \newenvironment{remark*}
  % {\leftskip6em\rightskip6em\par}
  % {\par}


%% Dummy environment for testing:

\newenvironment{foo}
  {\bf Foo.\ }
  {}


\newenvironment{foo*}
  {\bf Foo.\ }
  {}


\newenvironment{click}
  {\bf Click.\ }
  {}

\newenvironment{click*}
  {\bf Click.\ }
  {}


% \newenvironment{remark}
%   {\bf Remark.\ }
%   {}

\newenvironment{capsule}
  {\leftskip10em\par}
  {\par}

%%%%%%%%%%%%%%%%%%%%%%%%%%%%%%%%%%%%%%%%%%%%%%%%%%%%%



\title{Почему правильных многогранников всего пять?}

\begin{document}
\maketitle
\tableofcontents

\section{Правильные многогранники}

Правильный многогранник --- это выпуклый многогранник, все грани которого равные правильные многоугольники, а в каждой вершине сходится равное число рёбер. Всего существует 5 правильных многогранников:

\begin{enumerate}
  \item Тетраэдр
  \item Куб (гексаэдр)
  \item Октаэдр
  \item Додекаэдр
  \item Икосаэдр
\end{enumerate}


Почему их всего 5, и почему нет других, я объясню ниже.

\section{Почему их всего пять?}

Самый простой способ это выяснить --- посчитать сумму плоских углов при одной вершине. Если она меньше двух развёрнутых углов, то есть повод предполагать, что соответствующий многогранник существует.

Пусть грани многогранника правильные \(n\)-угольники, а у каждой из вершин сходится \(m\) рёбер. Пара \( (n, m)  \) называется символом Шлефли многогранника. Она полностью определяет многогранник.

Вернёмся к сумме углов при вершине. Сумма углов выпуклого \(n\)-угольника равна \( (n-2)\pi \), откуда угол в правильном \(n\)-угольнике равен
\[
    \alpha_n = (1-2/n)\pi.
\]
Домножив на \(m\), получим сумму углов, которая, как мы знаем, должна быть меньше \( 2\pi \):
\[
    m(1-2/n)\pi < 2\pi,
\]
\[
    \frac{1}{n} + \frac{1}{m} > \frac{1}{2}.
\]

Попробуем определить все возможные пары. Понятно, что 2-угольников в евклидовом пространстве не бывает, и по два ребра на вершину маловато будет --- иначе углы при этой вершине были бы развёрнутыми. С другой стороны, оба числа одновременно не могут быть больше трёх, в этом случае левая часть не больше половины. Поэтому остаются только следующие пары:
\begin{itemize}
  \item \( (3,3) \), соответствующая тетраэдру;
  \item \( (4,3) \) --- квадраты по 3 у вершины образуют куб;
  \item \( (3,4) \) описывает октаэдр;
  \item \( (5,3) \) додекаэдр
  \item \( (3,5) \) икосаэдр
\end{itemize}

Нетрудно заметить, что названия этим многогранникам даны по числу граней. Но как определить число вершин, рёбер и граней по символу Шлефли?

\section{Считаем вершины, рёбра и грани}

В этом нелёгком деле нам поможет теорема Эйлера для многогранников:
\[
    V - E + F = 2,
\]
где \(V\) --- число вершин, \(E\) --- рёбер, а \(F\) --- граней многогранника. Осталось связать эти числа друг с другом.
Заметим, что одно ребро принадлежит сразу двум граням. Каждая грань содержит n рёбер, поэтому
\[
    nF = 2E.
\]
Аналогично с вершинами: ребро соединяет 2 вершины, а в каждой вершине сходится по m рёбер, откуда
\[
    mV = 2E.
\]
Осталось выразить всё через число рёбер и подставить в теорему:
\[
    \frac{2E}{m} - E + \frac{2E}{n} = 2,
\]
\[
    \frac{1}{n} + \frac{1}{m} = \frac{1}{2} + \frac{1}{E}.
\]
Отсюда, кстати, следует неравенство из второй части. В итоге:
\[
\begin{array}{l}
    E = \frac{2mn}{2(n + m) - mn},\\
    V = \frac{4n}{2(n + m) - mn},\\
    F = \frac{4m}{2(n + m) - mn}.
\end{array}
\]

Подставляя в эту формулу значения, получаем 4 грани для тетраэдра, 6 для куба, 8 для октаэдра, 12 для додекаэдра и, наконец, 20 для икосаэдра.

\section{Конечные группы самосовмещений пространства}

Для полного счастья осталось приплести к этой проблеме теорию групп. Правильные многогранники тесно связаны с конечными группами самосовмещений пространства. Так как во всех вершинах сходится одинаковое число рёбер, а все грани --- равные правильные многоугольники, то вершины между собой никак не отличаются. Поэтому существует конечное число различных самосовмещений многогранников. Эти самосовмещения являются поворотами, оси которых проходят через центр симметрии многогранника. Эти оси могут проходить либо через середины двух рёбер (поворот на \(\pi\)), через пару вершин (на \(2\pi/m\)), через центры граней (на \(2\pi/n\)), и если числа \(n\) и \(m\) совпадают, то через вершину и центр противолежащей грани.

Так как последовательное применение любых двух самосовмещений правильного многогранника также является его самосовмещением, то они образуют конечную группу самосовмещений пространства. Отыщем все возможные такие группы.

Для начала, заметим, что самосовмещения пространства не исчерпываются поворотами. К ним также относятся параллельный перенос и винтовое перемещение, являющееся комбинацией поворота и параллельного переноса. Однако, если в группе есть перенос на вектор \(\vec{a}\), то там есть перенос на вектор \( n\vec{a} \), где \(n\) --- целое, то есть их бесконечно много. Иными словами, конечная группа не может содержать параллельного переноса, так как порождаемая им подгруппа параллельных переносов бесконечна. Вместе с параллельными переносами исчезают и винтовые перемещения.

Теперь перейдём к поворотам. Конечные группы порождаются лишь поворотами на углы вида \( 2\pi / n \). При этом оси всех поворотов должны пересекаться в одной точке, так как в противном случае комбинации поворотов могут не быть поворотами.

Если минимальный угол поворота вокруг оси составляет \( 2\pi/n \), то назовём число \(n\) порядком оси. Пусть группа порождается поворотами вокруг k осей, а порядок группы равен \(N\). Если ось имеет порядок \(n\), то существует \( \frac{N}{2n} \) положений, в которое она может переведена при преобразованиях, так как для каждого положения оси есть \(n\) поворотов вокруг неё и две возможных ориентации оси. Для каждого из положений есть \( n-1 \) нетождественный поворот. Суммируя по всем осям, получаем все повороты, за исключением тождественного:
\[
    \sum_{i=1}^k \frac{N}{2n_i}(n_i-1) = N - 1,
\]
\[
    \frac{k}{2} - \sum_{i=1}^k \frac{1}{2n_i} = 1 - \frac{1}{N},
\]
\[
    \sum_{i=1}^k \frac{1}{n_i} = k - 2 + \frac{2}{N}.
\]
Понятно, что \( N, n_i \ge 2 \), так как единичная группа нас не интересует. Начнём перебирать различные \( k \):
\begin{itemize}
  \item \(k=1\): \[ \frac{1}{n_1} = -1 + \frac{2}{N}, \]  правая часть не превышает нуля -- не подходит;
  \item \(k=2\): \[ \frac{1}{n_1} + \frac{1}{n_2} = \frac{2}{N}, \]  так как \( N \ge n_1, n_2 \), то возможен только случай \(N=n_1=n_2\). Это означает, что существует всего одна ось поворота, которую мы посчитали дважды. В этом случае искомая группа --- это группа вращения пирамиды (правильного многоугольника), циклическая группа порядка \( N \);
  \item \(k=3\): \[ \frac{1}{n_1} + \frac{1}{n_2} + \frac{1}{n_3} = 1 + \frac{2}{N}. \]  Заметим, что все три числа в левой части одновременно не могут превышать двух, поэтому одно из них, \(n_3\), положим равным 2. Тогда получаем уравнение \[ \frac{1}{n_1} + \frac{1}{n_2} = \frac{1}{2} + \frac{2}{N}, \] которое мы уже решали: пары \( (3,3) \), \( (4,3) \), \( (3,4) \), \( (5,3) \) и \( (3,5) \) соответствуют группам вращений правильных многогранников. При этом число \( n_1 \) определяет порядок оси, проходящей через центр грани, \( n_2 \) --- через вершину, а \(n_3 = 2\) --- через ребро.
Но это не все решения, есть ещё одно: \( n_1 = N/2,\ n_2=n_3=2 \) --- это группа диэдра. Диэдр получается из пирамиды, если к ней присоединить её отражение в основании;
  \item \(k=4\): \[ \frac{1}{n_1} + \frac{1}{n_2} + \frac{1}{n_3} + \frac{1}{n_4} = 2 + \frac{2}{N}, \]  левая часть не больше 2, правая строго больше -- не подходит.
  \item \(k \ge 5 \) также не подходит, так как с ростом k правая часть увеличивается быстрее левой.
\end{itemize}

Теперь вы знаете не менее 3 способов показать, почему правильных многогранников всего 5.
\end{document}