\documentclass[11pt,a4paper,russian,intlimits]{ncc}
\usepackage[utf8]{inputenc}
\usepackage[T2A]{fontenc}
\DeclareUnicodeCharacter{2009}{\,}
\usepackage[margin=2cm]{geometry}

\usepackage{color}

\usepackage{hyperref}
\hypersetup{
    colorlinks=true,
    linkcolor=black,
    filecolor=magenta,
    urlcolor=cyan,
    linktoc=all,
}

\usepackage{graphicx}

% Needed for Asciidoc

\newcommand{\admonition}[2]{\textbf{#1}: {#2}}
\newcommand{\rolered}[1]{ \textcolor{red}{#1} }
\newcommand{\roleblue}[1]{ \textcolor{blue}{#1} }


\renewenvironment{quotation}
{   \leftskip 4em \begin{em} }
{\end{em}\par }

\def\signed#1{{\leavevmode\unskip\nobreak\hfil\penalty50\hskip2em
  \hbox{}\nobreak\hfil\raise-3pt\hbox{(#1)}%
  \parfillskip=0pt \finalhyphendemerits=0 \endgraf}}


\newsavebox\mybox

\newenvironment{aquote}[1]
  {\savebox\mybox{#1}\begin{quotation}}
  {\signed{\usebox\mybox}\end{quotation}}

\newenvironment{tquote}[1]
  {  {\bf #1} \begin{quotation} \\ }
  { \end{quotation} }

%% BOXES: http://tex.stackexchange.com/questions/83930/what-are-the-different-kinds-of-boxes-in-latex
%% ENVIRONMENTS: https://www.sharelatex.com/learn/Environments

\newenvironment{asciidocbox}
  {\leftskip6em\rightskip6em\par}
  {\par}

\newenvironment{titledasciidocbox}[1]
  {\leftskip6em\rightskip6em\par{\bf #1}\vskip-0.6em\par}
  {\par}



%%%%%%%%%%%%%%%%%%%%%%%%%%%%%%%%%%%%%%%%%%%%%%%%%%%%%%%%

%% http://texblog.org/tag/rightskip/


\newenvironment{preamble}
  {}
  {}

%% http://tex.stackexchange.com/questions/99809/box-or-sidebar-for-additional-text
%%
\newenvironment{sidebar}[1][r]
  {\wrapfigure{#1}{0.5\textwidth}\tcolorbox}
  {\endtcolorbox\endwrapfigure}


%%%%%%%%%%

\newenvironment{comment*}
  {\leftskip6em\rightskip6em\par}
  {\par}

  % \newenvironment{remark*}
  % {\leftskip6em\rightskip6em\par}
  % {\par}


%% Dummy environment for testing:

\newenvironment{foo}
  {\bf Foo.\ }
  {}


\newenvironment{foo*}
  {\bf Foo.\ }
  {}


\newenvironment{click}
  {\bf Click.\ }
  {}

\newenvironment{click*}
  {\bf Click.\ }
  {}


% \newenvironment{remark}
%   {\bf Remark.\ }
%   {}

\newenvironment{capsule}
  {\leftskip10em\par}
  {\par}

%%%%%%%%%%%%%%%%%%%%%%%%%%%%%%%%%%%%%%%%%%%%%%%%%%%%%



\title{Классический осциллятор как квантовый}

\begin{document}
\maketitle

На примере гармонического осциллятора покажем, что обычный пружинный маятник может быть описан с точки зрения квантовой механики и полученные при этом результаты будут совпадать с классическими.

В квантовой механике для описания осциллятора используется волновая функция \(\Psi\), подчиняющаяся волновому уравнению:
\[
    i\hbar\frac{\partial \Psi}{\partial t} = \hat{H}\Psi,
\]
где
\[
    \hat{H} = \frac{\hat{p}^2}{2m} + \frac{k\hat{x}^2}{2}
\]
называется гамильтонианом нашего осциллятора. Операторы имеют различный вид в зависимости от представления. Рассмотрим координатное представление, в котором
\[
    \hat{x} = x,\quad \hat{p} = -i\hbar\frac{\partial}{\partial x}.
\]
В нём волновое уравнение примет вид
\[
    i\hbar\frac{\partial \Psi(x,t)}{\partial t} = \left( -\frac{\hbar^2}{2m}\frac{\partial^2}{\partial x^2} + \frac{kx^2}{2} \right)\Psi(x,t).
\]
Это уравнение можно решить методом разделения переменных:
\[
    \Psi(x,t) = \psi(x)\exp\left(-i\frac{E}{\hbar}t\right),
\]
где функция \(\psi(x)\) удовлетворяет дифференциальному уравнению
\[
    -\frac{h^2}{2m}\psi^{\prime\prime} + \left(\frac{kx^2}{2} - E\right)\psi = 0.
\]
Немного преобразуем его:
\[
    \psi^{\prime\prime} - \left(\frac{m^2\omega^2}{\hbar^2}x^2 - \frac{2mE}{\hbar^2}\right)\psi = 0.
\]
Перейдём к безразмерной переменной
\[
    y = \sqrt{\frac{m\omega}{\hbar}}x,
\]
относительно которой уравнение примет вид
\[
    \psi^{\prime\prime} - \left(y^2 - \frac{2E}{\hbar\omega}\right)\psi = 0.
\]
Выполним ещё одну подстановку:
\[
    \psi(y) = u(y)\exp(-y^2/2),
\]
\[
    u^{\prime\prime} - 2y u^\prime - (1 - y^2)u - (y^2 - \lambda)u = 0,
\]
\[
    u^{\prime\prime} - 2y u^\prime + (\lambda - 1)u = 0.
\]
Это уравнение полиномов Эрмита, если положить \( \lambda - 1 = 2n \).
Таким образом, энергия осциллятора может принимать только значения
\[
    E_n = \frac{\hbar\omega}{2} + n\hbar\omega,\ n=0,1,2,\ldots
\]
а волновая функция при этом будет иметь вид
\[
    \Psi_n(x, t) = C\tilde{H}_n\left(\sqrt{\frac{m\omega}{\hbar}}x\right)\exp\left(-\frac{m\omega x^2}{2\hbar}\right)\exp\left(-i\frac{E_n}{\hbar}t\right),
\]
где \( \tilde{H}_n \) -- нормированные многочлены Эрмита, а \( C \) -- нормировочный коэффициент, который определяется из условия нормировки
\[
    \int\limits_{-\infty}^{+\infty} \Psi_n^\star  \Psi_n\,dx = 1 \Rightarrow
    C = \sqrt[4]{\frac{m\omega}{\hbar}}.
\]

Найденные нами состояния характеризуются неизменным во времени значением энергии. Такие состояния называются стационарными. Посмотрим, как в таких состояниях меняется положение маятника:
\[
    \langle x_n(t) \rangle = \langle \Psi_n | x | \Psi_n \rangle = \int\limits_{-\infty}^{+\infty}
    \Psi_n^\star  x \Psi_n\, dx = 0.
\]
То есть маятник, обладая некоторой энергией, находится в положении равновесия. Тут мы видим некоторое расхождение с поведением обычного маятника. Но это расхождение довольно легко объясняется: для возбуждения стационарного состояния осциллятору нужно передать энергию, равную энергии этого состояния, что в случае маятника невозможно сделать ввиду малой разности между уровнями.

Чтобы с этой точки зрения объяснить поведение обычного маятника, требуется возбудить сразу несколько стационарных состояний:
\[
    \Psi = \sum c_n \Psi_n,\quad \sum |c_n|^2 = 1.
\]
Энергия такого состояния
\[
    E = \sum c_m^\star c_n \langle\Psi_m|\hat{H}|\Psi_n\rangle = \sum c_m^\star c_n E_n\langle\Psi_m|\Psi_n\rangle = |c_n|^2 E_n,
\]
а координата будет изменяться следующим образом:
\[
    x(t) = \sum c_m^\star c_n \langle\Psi_m|x|\Psi_n\rangle = C^2\sum c_m^\star c_n \langle \tilde{H}_m(y) | x e^{-y^2} | \tilde{H}_n(y)\rangle \exp\left(-i\frac{E_n-E_m}{\hbar}t\right).
\]
Рассмотрим подробнее величину
\[
    \langle \tilde{H}_m(y) | xe^{-y^2} | \tilde{H}_n(y)\rangle = \int\limits_{-\infty}^{+\infty} e^{-y^2}\tilde{H}_m^\star (y) x \tilde{H}_n(y)\, dx = \frac{\hbar}{m\omega} \int\limits_{-\infty}^{+\infty} e^{-y^2} y \tilde{H}_m(y) \tilde{H}_n(y)\, dy.
\]
Так как многочлены Эрмита \( \tilde{H}_m(y) \) ортогональны на действительной прямой с весом \( e^{-y^2} \), то это скалярное произведение будет отлично от нуля только в случае \( |n - m| = 1 \):
\[
    \langle \tilde{H}_m(y) | xe^{-y^2} | \tilde{H}_n(y)\rangle =
    \frac{\hbar}{m\omega}(a_n\delta_{n, m+1} + a_m\delta_{m, n+1}),
\]
где
\[
    a_n = \int\limits_{-\infty}^{+\infty} e^{-y^2} y \tilde{H}_n(y) \tilde{H}_{n-1}(y)\, dy = \sqrt{\frac{n}{2}}.
\]
Возвращаясь, получаем
\[
    x(t) = \sqrt{\frac{\hbar}{m\omega}}\sum c_m^\star c_n (a_n\delta_{n, m+1} + a_m\delta_{m, n+1}) \exp(-i(n-m)\omega t),
\]
\[
    x(t) = \sqrt{\frac{\hbar}{m\omega}}\sum  a_n (c_n c_{n-1}^\star  e^{-i\omega t}+ c_{n-1} c_n^\star  e^{i\omega t}) = 
    2\sqrt{\frac{\hbar}{m\omega}}S  \cos(\omega t + \phi_0),
\]
где
\[
    S = \left|\sum a_n c_n^\star  c_{n-1}\right|,\quad
    \phi_0 = \arg \sum a_n c_n^\star  c_{n-1}.
\]

Пусть энергия осциллятора равна \( E \) и она находится вблизи уровня \( E_n = \hbar\omega(n+1/2) \). Положим, что коэффициенты состояний с близкими к среднему значению энергиями много больше остальных. Пусть вблизи состояния с номером \( n \) заметные амплитуды имеют \( 2n_1 \) состояний, причём \( n \gg n_1 \gg 1 \) и \( c_{n-n_1} \approx \ldots \approx c_n \approx \ldots \approx c_{n+n_1} \). Тогда
\[
    \sum_{m=n-n_1}^{n+n_1} c_m^\star  c_{m} \approx 1,
\]
\[
    S \approx \left|\sum_{m=n-n_1+1}^{n+n_1} a_m c_m^\star  c_{m-1}\right| \approx \left|\sum_{m=n-n_1+1}^{n+n_1} \sqrt{\frac{m}{2}}c_m c_m^\star \right| \approx \sqrt{\frac{n}{2}}.
\]
Амплитуда колебаний получается равной
\[
    A = 2\sqrt{\frac{n\hbar\omega}{2m\omega^2}} = 2\sqrt{\frac{E}{2k}},
\]
откуда получается классическое
\[
    E = \frac{kA^2}{2}.
\]
Таким образом, энергия колебаний маятника связана с их амплитудой знакомой со школы формулой.

\end{document}