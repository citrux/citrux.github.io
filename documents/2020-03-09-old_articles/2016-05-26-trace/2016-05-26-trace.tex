\documentclass{ncc}
\usepackage[utf8]{inputenc}
\usepackage[russian]{babel}
\usepackage[T2A]{fontenc}
\usepackage{amssymb}
\usepackage{amsmath}
\usepackage{pscyr}
\usepackage{graphicx}
\usepackage{listings}
\usepackage[colorlinks,linkcolor=black,urlcolor=blue]{hyperref}

\lstloadlanguages{C++}
\lstset{
language=C++,
extendedchars=\true, %Чтобы русские буквы в комментариях были
keepspaces=true,
inputencoding=utf8,
breaklines,
columns=fullflexible,
flexiblecolumns,
numbers=left,
numberstyle={\footnotesize},
commentstyle=\it,
stringstyle=\bf,
belowcaptionskip=5pt }


\title{Инварианты. След тензора}

\begin{document}
\maketitle

Давно я сюда ничего не писал, надо разбавить это неловкое пятимесячное молчание. Разбавить чем-то простеньким, но со вкусом.

На днях мой товарищ Серёжа спросил у меня: «А с чего бы это вдруг след тензора 2 ранга является инвариантом преобразования координат? Все об этом говорят, но доказательства я не видел.» Ну мы с ним на скорую руку обрисовали кривенькое доказательство, но меня оно не очень-то устроило. И придумал я куда менее кривое доказательство.

Итак, пусть имеется тензор 2 ранга \( A \), который в собственном базисе имеет диагональный вид
\[
    a_{ij} = \lambda_i\delta_{ij},
\]
а его след
\[
    \mathrm{Tr} A = \sum_i\lambda_i.
\]

Рассмотрим теперь произвольное преобразование координат, определяемое матрицей \( T \). При этом преобразовании \( A^\prime = TAT^{-1} \) и компоненты тензора будут иметь вид
\[
    a^\prime_{ij} = \sum_{k,l} t_{ik}a_{kl}t^{-1}_{lj} = \sum_{k,l} t_{ik}\lambda_k\delta_{kl}t^{-1}_{lj} = \sum_{k} \lambda_kt_{ik}t^{-1}_{kj},
\]
а след
\[
    \mathrm{Tr} A^\prime = \sum_{i,j}\sum_k \lambda_kt_{ik}t^{-1}_{kj}\delta_{ij} = \sum_k \lambda_k \sum_i t^{-1}_{ki} t_{ik} =\sum_k \lambda_k \delta_{kk} = \sum_k \lambda_k = \mathrm{Tr} A.
\]

Что и требовалось доказать.
\end{document}
