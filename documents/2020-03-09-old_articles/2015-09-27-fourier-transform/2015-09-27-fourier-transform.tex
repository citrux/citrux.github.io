\documentclass{ncc}
\usepackage[utf8]{inputenc}
\usepackage[russian]{babel}
\usepackage[T2A]{fontenc}
\usepackage{amssymb}
\usepackage{amsmath}
\usepackage{pscyr}
\usepackage{graphicx}
\usepackage{listings}
\usepackage[colorlinks,linkcolor=black,urlcolor=blue]{hyperref}

\lstloadlanguages{C++}
\lstset{
language=C++,
extendedchars=\true, %Чтобы русские буквы в комментариях были
keepspaces=true,
inputencoding=utf8,
breaklines,
columns=fullflexible,
flexiblecolumns,
numbers=left,
numberstyle={\footnotesize},
commentstyle=\it,
stringstyle=\bf,
belowcaptionskip=5pt }


\title{Постигая преобразование Фурье}

\begin{document}
\maketitle
\tableofcontents
\section{Ряд Фурье и преобразование Фурье}
Рассмотрим непрерывную периодическую функцию \( f \) с периодом \( 2\pi \).
Предположим, что её можно представить в виде ряда:
\[
    f(x) = \sum\limits_{n=-\infty}^{+\infty} c_n e^{inx}.
\]
Заметим, что
\[
    \int\limits_0^{2\pi} e^{i(n-m)x} dx = 2\pi\delta_{mn},
\]
откуда получим простой способ определения коэффициентов \( c_n \):
\[
    \int\limits_0^{2\pi} f(x) e^{-inx} dx = 2\pi c_n,
\]
\[
    c_n = \frac{1}{2\pi}\int\limits_0^{2\pi} f(x) e^{-inx} dx.
\]
Такой ряд называется рядом Фурье функции \( f \).

Если период функции отличается от \( 2\pi \), то можно сделать замену
переменной и получить
\[
    f(x) = \sum\limits_{n=0}^\infty c_n e^\frac{in2\pi x}{L},\quad
    c_n = \frac{1}{L}\int\limits_{-L/2}^{L/2} f(x^\prime) e^\frac{-in2\pi x^\prime}{L} dx^\prime.
\]
Для непериодических функций растянем период на всю прямую \( \mathbb{R} \)
\[
   c_n = \lim_{L\to\infty} \frac{1}{L}\int\limits_{-L/2}^{L/2} f(x^\prime) e^\frac{-in2\pi x^\prime}{L} dx^\prime.
\]
Пусть \( k_n = 2\pi n / L,\ \Delta k = 2\pi / L \):
\[
   c_n = \lim_{\Delta k \to 0} \frac{\Delta k}{2\pi}\int\limits_{-\infty}^{+\infty} f(x^\prime) e^{-ik_nx^\prime} dx^\prime.
\]
Вернёмся к ряду:
\[
    f(x) = \lim_{\Delta k\to 0}\frac{1}{2\pi}\sum\limits_{n=-\infty}^{+\infty} \Delta k
    e^{ik_nx}\int\limits_{-\infty}^{+\infty} f(x^\prime) e^{-ik_nx^\prime}
    dx^\prime.
\]
Нетрудно понять, что предел суммы -- это интеграл, поэтому
\[
    f(x) = \frac{1}{\sqrt{2\pi}}\int\limits_{-\infty}^{+\infty}  \left[\frac{1}{\sqrt{2\pi}}\int\limits_{-\infty}^{+\infty} f(x^\prime) e^{-ikx^\prime} dx^\prime\right] e^{ikx} dk.
\]
Преобразованием Фурье называется отображение
\[
    \tilde{f}(k) = \frac{1}{\sqrt{2\pi}}\int\limits_{-\infty}^{+\infty} f(x) e^{-ikx} dx.
\]
Оно позволяет по «форме сигнала» \(f(x)\) судить о его спектре
\(\tilde{f}(k)\). А чтобы собрать сигнал из его спектра пользуются обратным
преобразованием Фурье:
\[
    f(x) = \frac{1}{\sqrt{2\pi}}\int\limits_{-\infty}^{+\infty} \tilde{f}(k) e^{ikx} dk.
\]

В дальнейшем нам понадобится понятие \textbf{скалярного произведения функций}, поэтому
для начала разберём его.

\section{Скалярное произведение функций}

Для векторов понятие скалярного произведения вводится обычно следующим
образом
\[
    \vec{a} \cdot \vec{b} = (\vec{a}, \vec{b}) = |\vec{a}| |\vec{b}| \cos\theta_{ab}.
\]
В ортонормированном базисе его можно записать в виде
\[
    (\vec{a}, \vec{b}) = a_ib_i,
\]

Для функций можно ввести операцию, свойства которой будут очень похожи на
свойства скалярного произведения векторов:
\[
    \langle f | g \rangle = \int\limits_{-\infty}^{+\infty} f^\ast(x) g(x)\,
    dx.
\]

Заметим, что в этом контексте преобразование Фурье выглядит как матрица
перехода от одного базиса к другому. Рассмотрим его действие на
скалярное произведение.

\section{Равенство Парсеваля}

Представим в скалярном произведении двух функций одну из них через её
Фурье-образ:
\[
    \langle f | g \rangle =
    \frac{1}{\sqrt{2\pi}}\iint\limits_{-\infty}^{+\infty}
    \tilde{f}^\ast(k) e^{-ikx} g(x) \,dx\,dk=
    \int\limits_{-\infty}^{+\infty} \tilde{f}^\ast(k) \tilde{g}(k)\,dk =
    \langle \tilde{f} | \tilde{g} \rangle.
\]
Скалярное произведение -- инвариант преобразования Фурье. В этом
прослеживается аналогия с обычными векторами, скалярное произведение которых
является инвариантом преобразования базиса.

\section{Соотношение неопределённостей}
Один из фундаментальных результатов квантовой механики -- соотношение
неопределённостей
\[
    \Delta x  \Delta p_x \ge \frac{\hbar}{2}
\]
является свойством преобразования Фурье. Под неопределённостями здесь понимают
среднеквадратичные отклонения сопряжённых координаты и импульса:
\[
    \Delta x = \sqrt{\langle(x-\langle x \rangle)^2\rangle},\quad
    \Delta p_x = \hbar\sqrt{\langle(k-\langle k \rangle)^2\rangle}.
\]
Так как с корнями работать неудобно, перейдём к дисперсиям:
\[
    D(x)\cdot D(k) = \langle\psi(x)|(x-\langle x\rangle)^2|\psi(x)\rangle
    \langle\tilde{\psi}(k)|(k-\langle k \rangle)^2|\tilde{\psi}(k)\rangle.
\]
Здесь функции \( \psi \)  и \( \tilde\psi \) положим нормированными на 1.
Для удобства выполним преобразование координат:
\[
    x^\prime = x - \langle x \rangle,\quad
    k^\prime = k - \langle k \rangle,
\]
и рассмотрим вспомогательную функцию
\[
    \phi(x^\prime) = e^{-i\langle k \rangle x^\prime} \psi(x),\quad
    \tilde{\phi}(k^\prime) = e^{ik\langle x \rangle} \tilde{\psi}(k),
\]
используя которую перепишем искомое произведение в виде
\[
    D(x)\cdot D(k) = \int\limits_{-\infty}^{+\infty} x^2 \phi^\ast(x)\phi(x)
    \,dx\cdot \int\limits_{-\infty}^{+\infty} k^2 \tilde{\phi}^\ast(k)\tilde{\phi}(k)\,dk
\]
или
\[
    D(x)\cdot D(k) = \langle x\phi|x \phi\rangle
    \langle ik\tilde{\phi}|ik \tilde{\phi}\rangle =
    \langle x\phi|x \phi\rangle
    \langle \tilde{\left(\phi^\prime\right)}|\tilde{\left(\phi^\prime\right)}\rangle,
\]
где неявно используется равенство
\[
    \tilde{(\phi^\prime)} = \frac{1}{\sqrt{2\pi}} \int\limits_{-\infty}^{+\infty}
    \phi^\prime e^{-ikx}\,dx =
    \left.\frac{\phi e^{-ikx}}{\sqrt{2\pi}}\right|_{-\infty}^{+\infty}
    +  ik\tilde{\phi} = ik\tilde{\phi}.
\]
Пользуясь равенством Парсеваля, получаем
\[
    D(x)\cdot D(k) = \langle x \phi|x \phi\rangle \langle \phi^\prime | \phi^\prime \rangle.
\]
Это ни что иное, как произведение квадратов двух норм. Используя неравенство
Коши-Буняковского, получаем:
\[
    \langle x\phi|x \phi\rangle
    \langle \phi^\prime|\phi^\prime\rangle \ge
    \left|\langle x\phi | \phi^\prime \rangle\right|^2.
\]
Рассмотрим внимательнее правую часть и проинтегрируем по частям:
\[
    \langle x\phi|\phi^\prime\rangle = -\langle \phi^\prime | x\phi \rangle
    - \langle\phi|\phi\rangle,
\]
\[
    \langle x\phi|\phi^\prime\rangle + \langle\phi^\prime|x\phi\rangle = -1.
\]
Воспользуемся неравенством треугольника:
\[
    |\langle x\phi|\phi^\prime\rangle| + |\langle\phi^\prime|x\phi\rangle| \ge 1,
    \quad
    |\langle x\phi|\phi^\prime\rangle| \ge \frac{1}{2}.
\]
Возвращаясь к неравенству Коши-Буняковского, имеем
\[
    D(x)\cdot D(k) \ge \frac{1}{4},
\]
откуда получаем соотношение неопределённостей:
\[
    \Delta x  \Delta p_x \ge \frac{\hbar}{2}.
\]

Совершенно аналогично можно доказать другое соотношение:
\[
    \Delta E  \Delta t \ge \frac{\hbar}{2}.
\]

На этом пока всё.
\end{document}
