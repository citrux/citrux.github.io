\documentclass{ncc}
\usepackage[utf8]{inputenc}
\usepackage[russian]{babel}
\usepackage[T2A]{fontenc}
\usepackage{amssymb}
\usepackage{amsmath}
\usepackage{pscyr}
\usepackage{graphicx}
\usepackage{listings}
\usepackage[colorlinks,linkcolor=black,urlcolor=blue]{hyperref}

\lstloadlanguages{C++}
\lstset{
language=C++,
extendedchars=\true, %Чтобы русские буквы в комментариях были
keepspaces=true,
inputencoding=utf8,
breaklines,
columns=fullflexible,
flexiblecolumns,
numbers=left,
numberstyle={\footnotesize},
commentstyle=\it,
stringstyle=\bf,
belowcaptionskip=5pt }

\title{Rvalue-reference в D}
\begin{document}
\maketitle

Речь сегодня пойдёт о передаче в функцию rvalue по ссылке. Сравнивать будем C++ и D.

Как правило, при передаче в функцию структуры желательно избегать лишнего копирования, так как это сказывается на производительности не в лучшую сторону. Для этого используется передача по ссылке. В случае C++
\begin{verbatim}
#include <iostream>
#include <cmath>

using namespace std;

struct vec {
  double x, y, z;
};

double length(vec& v) {
  return sqrt(v.x * v.x + v.y * v.y + v.z * v.z);
}

int main() {
  vec a = {3, 4, 12};
  cout << length(a) << endl;
  return 0;
}
\end{verbatim}
Для D всё выглядит очень похоже:
\begin{verbatim}
import std.stdio;
import std.math;

struct vec {
  double x, y, z;
};

double length(ref vec v) {
  return sqrt(v.x * v.x + v.y * v.y + v.z * v.z);
}

void main() {
  vec a = {3, 4, 12};
  writeln(length(a));
}
\end{verbatim}
Но что, если мы не захотим создавать переменную \texttt{a}, а попробуем передать в функцию в качестве аргумента конструктор:
\begin{verbatim}
length(vec(3,4,12))
\end{verbatim}
В этом случае оба компилятора будут ругаться на то, что они не могут получить неконстантную ссылку на rvalue, коим является результат вызова конструктора. Функция \texttt{length} не изменяет своего аргумента, поэтому слегка её изменим, сделав аргумент константной ссылкой:
\begin{verbatim}
double length(const vec& v) {
  return sqrt(v.x * v.x + v.y * v.y + v.z * v.z);
}
\end{verbatim}
На D это выглядит так:
\begin{verbatim}
double length(ref const vec v) {
  return sqrt(v.x * v.x + v.y * v.y + v.z * v.z);
}
\end{verbatim}
В С++ это решает проблему, а в D нет -- он считает, что реальный тип аргумента vec, а ожидаемый \texttt{ref const vec}. Дело в том, что D воспринимает rvalue только как значение, но при передаче в функцию лишнего копирования не происходит. Поэтому от нас требуется определить такую функцию, которая сможет одновременно принимать и \texttt{ref const vec}, и просто \texttt{vec}. И тут не обойтись без шаблона:
\begin{verbatim}
double length(T: vec)(auto ref const T v) {
  return sqrt(v.x * v.x + v.y * v.y + v.z * v.z);
}
\end{verbatim}

Собирая всё в кучу, имеем
\begin{verbatim}
import std.stdio;
import std.math;

struct vec {
  double x, y, z;
};

double length(T: vec)(auto ref const T v) {
  return sqrt(v.x * v.x + v.y * v.y + v.z * v.z);
}

void main() {
  vec a = {3, 4, 12};
  writeln(length(a));
}
\end{verbatim}

Можно заметить, что проблемы можно избежать, определив \texttt{length} как метод \texttt{vec}:
\begin{verbatim}
import std.stdio;
import std.math;

struct vec {
  double x, y, z;

  double length() {
    return sqrt(this.x * this.x + this.y * this.y + this.z * this.z);
  }
};

void main() {
  vec a = {3, 4, 12};
  writeln(length(a));
}
\end{verbatim}
Но если нам понадобится скалярное произведение, то всё равно придётся писать такую вот шаблонную функцию.

В общем, если в D вам потребуется аналог \texttt{const~T\&}, то это \texttt{auto~ref~const~T}.
\end{document}
