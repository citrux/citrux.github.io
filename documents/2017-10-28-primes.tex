\documentclass[11pt,a4paper,russian,intlimits]{ncc}
\usepackage[utf8]{inputenc}
\usepackage[T2A]{fontenc}
\DeclareUnicodeCharacter{2009}{\,}
\usepackage[margin=2cm]{geometry}

\usepackage{color}

\usepackage{hyperref}
\hypersetup{
    colorlinks=true,
    linkcolor=black,
    filecolor=magenta,
    urlcolor=cyan,
    linktoc=all,
}

\usepackage{graphicx}

% Needed for Asciidoc

\newcommand{\admonition}[2]{\textbf{#1}: {#2}}
\newcommand{\rolered}[1]{ \textcolor{red}{#1} }
\newcommand{\roleblue}[1]{ \textcolor{blue}{#1} }


\renewenvironment{quotation}
{   \leftskip 4em \begin{em} }
{\end{em}\par }

\def\signed#1{{\leavevmode\unskip\nobreak\hfil\penalty50\hskip2em
  \hbox{}\nobreak\hfil\raise-3pt\hbox{(#1)}%
  \parfillskip=0pt \finalhyphendemerits=0 \endgraf}}


\newsavebox\mybox

\newenvironment{aquote}[1]
  {\savebox\mybox{#1}\begin{quotation}}
  {\signed{\usebox\mybox}\end{quotation}}

\newenvironment{tquote}[1]
  {  {\bf #1} \begin{quotation} \\ }
  { \end{quotation} }

%% BOXES: http://tex.stackexchange.com/questions/83930/what-are-the-different-kinds-of-boxes-in-latex
%% ENVIRONMENTS: https://www.sharelatex.com/learn/Environments

\newenvironment{asciidocbox}
  {\leftskip6em\rightskip6em\par}
  {\par}

\newenvironment{titledasciidocbox}[1]
  {\leftskip6em\rightskip6em\par{\bf #1}\vskip-0.6em\par}
  {\par}



%%%%%%%%%%%%%%%%%%%%%%%%%%%%%%%%%%%%%%%%%%%%%%%%%%%%%%%%

%% http://texblog.org/tag/rightskip/


\newenvironment{preamble}
  {}
  {}

%% http://tex.stackexchange.com/questions/99809/box-or-sidebar-for-additional-text
%%
\newenvironment{sidebar}[1][r]
  {\wrapfigure{#1}{0.5\textwidth}\tcolorbox}
  {\endtcolorbox\endwrapfigure}


%%%%%%%%%%

\newenvironment{comment*}
  {\leftskip6em\rightskip6em\par}
  {\par}

  % \newenvironment{remark*}
  % {\leftskip6em\rightskip6em\par}
  % {\par}


%% Dummy environment for testing:

\newenvironment{foo}
  {\bf Foo.\ }
  {}


\newenvironment{foo*}
  {\bf Foo.\ }
  {}


\newenvironment{click}
  {\bf Click.\ }
  {}

\newenvironment{click*}
  {\bf Click.\ }
  {}


% \newenvironment{remark}
%   {\bf Remark.\ }
%   {}

\newenvironment{capsule}
  {\leftskip10em\par}
  {\par}

%%%%%%%%%%%%%%%%%%%%%%%%%%%%%%%%%%%%%%%%%%%%%%%%%%%%%


\title{Гармонический ряд по простым числам}
\begin{document}
\maketitle
Покажем, что ряд
\begin{equation*}
    \sum_{p-\text{простое}}\frac{1}{p}
\end{equation*}
расходится. Для этого воспользуемся дзета-функцией Римана и её представлением в виде бесконечного произведения:
\begin{equation*}
    \zeta(s) = \prod_{p-\text{простое}}\frac{1}{1-p^{-s}}.
\end{equation*}
Рассмотрим логарифм правой части при \( s = 1 \):
\begin{equation*}
    \ln\prod_{p-\text{простое}}\frac{1}{1-p^{-s}} = -\sum_{p-\text{простое}}\ln\left(1-p^{-s}\right) = \sum_{p-\text{простое}}\frac{1}{p} + \sum_{n=2}^\infty\sum_{p-\text{простое}}\frac{1}{np^n}
\end{equation*}
Последнее слагаемое можно оценить сверху:
\begin{equation*}
    \sum_{n=2}^\infty\sum_{p-\text{простое}}\frac{1}{np^n} <
    \sum_{n=2}^\infty\sum_{k=2}^\infty\frac{1}{nk^n} <
    \sum_{k=2}^\infty\sum_{n=2}^\infty\frac{1}{k^n} =
    \sum_{k=2}^\infty\frac{1}{k(k-1)} = 1.
\end{equation*}
Таким образом,
\begin{equation*}
    \ln\zeta(1) < 1 + \sum_{p-\text{простое}}\frac{1}{p},
\end{equation*}
\begin{equation*}
    \sum_{p-\text{простое}}\frac{1}{p} > \ln\zeta(1) - 1.
\end{equation*}
Но \( \zeta(1) \) равна сумме гармонического ряда, поэтому ряд
\begin{equation*}
    \sum_{p-\text{простое}}\frac{1}{p}
\end{equation*}
расходится.
\end{document}