\documentclass{ncc}
\usepackage[utf8]{inputenc}
\usepackage[russian]{babel}
\usepackage[T2A]{fontenc}
\usepackage{amssymb}
\usepackage{amsmath}
\usepackage{pscyr}
\usepackage{graphicx}
\usepackage{listings}
\usepackage[colorlinks,linkcolor=black,urlcolor=blue]{hyperref}

\lstloadlanguages{C++}
\lstset{
language=C++,
extendedchars=\true, %Чтобы русские буквы в комментариях были
keepspaces=true,
inputencoding=utf8,
breaklines,
columns=fullflexible,
flexiblecolumns,
numbers=left,
numberstyle={\footnotesize},
commentstyle=\it,
stringstyle=\bf,
belowcaptionskip=5pt }

\title{Гармонический ряд по простым числам}
\begin{document}
\maketitle
Покажем, что ряд
\begin{equation*}
    \sum_{p-\text{простое}}\frac{1}{p}
\end{equation*}
расходится. Для этого воспользуемся дзета-функцией Римана и её представлением в виде бесконечного произведения:
\begin{equation*}
    \zeta(s) = \prod_{p-\text{простое}}\frac{1}{1-p^{-s}}.
\end{equation*}
Рассмотрим логарифм правой части при \( s = 1 \):
\begin{equation*}
    \ln\prod_{p-\text{простое}}\frac{1}{1-p^{-s}} = -\sum_{p-\text{простое}}\ln\left(1-p^{-s}\right) = \sum_{p-\text{простое}}\frac{1}{p} + \sum_{n=2}^\infty\sum_{p-\text{простое}}\frac{1}{np^n}
\end{equation*}
Последнее слагаемое можно оценить сверху:
\begin{equation*}
    \sum_{n=2}^\infty\sum_{p-\text{простое}}\frac{1}{np^n} <
    \sum_{n=2}^\infty\sum_{k=2}^\infty\frac{1}{nk^n} <
    \sum_{k=2}^\infty\sum_{n=2}^\infty\frac{1}{k^n} =
    \sum_{k=2}^\infty\frac{1}{k(k-1)} = 1.
\end{equation*}
Таким образом,
\begin{equation*}
    \ln\zeta(1) < 1 + \sum_{p-\text{простое}}\frac{1}{p},
\end{equation*}
\begin{equation*}
    \sum_{p-\text{простое}}\frac{1}{p} > \ln\zeta(1) - 1.
\end{equation*}
Но \( \zeta(1) \) равна сумме гармонического ряда, поэтому ряд
\begin{equation*}
    \sum_{p-\text{простое}}\frac{1}{p}
\end{equation*}
расходится.
\end{document}