\documentclass[11pt,a4paper,russian,intlimits]{ncc}
\usepackage[utf8]{inputenc}
\usepackage[T2A]{fontenc}
\DeclareUnicodeCharacter{2009}{\,}
\usepackage[margin=2cm]{geometry}

\usepackage{color}

\usepackage{hyperref}
\hypersetup{
    colorlinks=true,
    linkcolor=black,
    filecolor=magenta,
    urlcolor=cyan,
    linktoc=all,
}

\usepackage{graphicx}

% Needed for Asciidoc

\newcommand{\admonition}[2]{\textbf{#1}: {#2}}
\newcommand{\rolered}[1]{ \textcolor{red}{#1} }
\newcommand{\roleblue}[1]{ \textcolor{blue}{#1} }


\renewenvironment{quotation}
{   \leftskip 4em \begin{em} }
{\end{em}\par }

\def\signed#1{{\leavevmode\unskip\nobreak\hfil\penalty50\hskip2em
  \hbox{}\nobreak\hfil\raise-3pt\hbox{(#1)}%
  \parfillskip=0pt \finalhyphendemerits=0 \endgraf}}


\newsavebox\mybox

\newenvironment{aquote}[1]
  {\savebox\mybox{#1}\begin{quotation}}
  {\signed{\usebox\mybox}\end{quotation}}

\newenvironment{tquote}[1]
  {  {\bf #1} \begin{quotation} \\ }
  { \end{quotation} }

%% BOXES: http://tex.stackexchange.com/questions/83930/what-are-the-different-kinds-of-boxes-in-latex
%% ENVIRONMENTS: https://www.sharelatex.com/learn/Environments

\newenvironment{asciidocbox}
  {\leftskip6em\rightskip6em\par}
  {\par}

\newenvironment{titledasciidocbox}[1]
  {\leftskip6em\rightskip6em\par{\bf #1}\vskip-0.6em\par}
  {\par}



%%%%%%%%%%%%%%%%%%%%%%%%%%%%%%%%%%%%%%%%%%%%%%%%%%%%%%%%

%% http://texblog.org/tag/rightskip/


\newenvironment{preamble}
  {}
  {}

%% http://tex.stackexchange.com/questions/99809/box-or-sidebar-for-additional-text
%%
\newenvironment{sidebar}[1][r]
  {\wrapfigure{#1}{0.5\textwidth}\tcolorbox}
  {\endtcolorbox\endwrapfigure}


%%%%%%%%%%

\newenvironment{comment*}
  {\leftskip6em\rightskip6em\par}
  {\par}

  % \newenvironment{remark*}
  % {\leftskip6em\rightskip6em\par}
  % {\par}


%% Dummy environment for testing:

\newenvironment{foo}
  {\bf Foo.\ }
  {}


\newenvironment{foo*}
  {\bf Foo.\ }
  {}


\newenvironment{click}
  {\bf Click.\ }
  {}

\newenvironment{click*}
  {\bf Click.\ }
  {}


% \newenvironment{remark}
%   {\bf Remark.\ }
%   {}

\newenvironment{capsule}
  {\leftskip10em\par}
  {\par}

%%%%%%%%%%%%%%%%%%%%%%%%%%%%%%%%%%%%%%%%%%%%%%%%%%%%%



\title{Инварианты. След тензора}

\begin{document}
\maketitle

Давно я сюда ничего не писал, надо разбавить это неловкое пятимесячное молчание. Разбавить чем-то простеньким, но со вкусом.

На днях мой товарищ Серёжа спросил у меня: «А с чего бы это вдруг след тензора 2 ранга является инвариантом преобразования координат? Все об этом говорят, но доказательства я не видел.» Ну мы с ним на скорую руку обрисовали кривенькое доказательство, но меня оно не очень-то устроило. И придумал я куда менее кривое доказательство.

Итак, пусть имеется тензор 2 ранга \( A \), который в собственном базисе имеет диагональный вид
\[
    a_{ij} = \lambda_i\delta_{ij},
\]
а его след
\[
    \mathrm{Tr} A = \sum_i\lambda_i.
\]

Рассмотрим теперь произвольное преобразование координат, определяемое матрицей \( T \). При этом преобразовании \( A^\prime = TAT^{-1} \) и компоненты тензора будут иметь вид
\[
    a^\prime_{ij} = \sum_{k,l} t_{ik}a_{kl}t^{-1}_{lj} = \sum_{k,l} t_{ik}\lambda_k\delta_{kl}t^{-1}_{lj} = \sum_{k} \lambda_kt_{ik}t^{-1}_{kj},
\]
а след
\[
    \mathrm{Tr} A^\prime = \sum_{i,j}\sum_k \lambda_kt_{ik}t^{-1}_{kj}\delta_{ij} = \sum_k \lambda_k \sum_i t^{-1}_{ki} t_{ik} =\sum_k \lambda_k \delta_{kk} = \sum_k \lambda_k = \mathrm{Tr} A.
\]

Что и требовалось доказать.
\end{document}